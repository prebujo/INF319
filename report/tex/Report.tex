\documentclass[a4paper,10pt]{article}
\usepackage[usenames, dvipsnames]{color}
\usepackage{amsmath}
\usepackage[sort,comma]{natbib}
\usepackage{geometry}
\usepackage{cancel}
\begin{document}

\title{pickup \& delivery problem with mulitiple time windows}
\author{preben bucher-johannessen}
\date{\today}
\maketitle

\pagenumbering{roman}
\tableofcontents
\newpage
\pagenumbering{arabic}

\section*{introduction}
the automobile industry is one of the most significant industries in Europe today which is facing high cost-reduction pressure from fierce international competition and difficult conditions.\cite{4flowWeb}
The supply chain process in automobile industry requires alot of flexibility in their planning and modelling of transportation which often takes the form some sort of vehicle transportation problem.
In planning the most efficient way of organizing your transportation network the classical VRPTW from is likely the first problem that comes to mind.

\cite{parragh08}
The Vehicle Routing Problem with Time Window Constraints (VRPTW) is defined as the problem of minimizing costs when a fleet of homogeneus vehicles has to destribute goods from a depot to a set of customers satisfying time windows and capacity constraints \cite{cordeau00}. 
Another classical problem is the PDPTW, Pickup and Delivery Problem with Time Windows, with multiple vehicles is another problem that has been widely researched and solved for example in \cite{nanry00}. 
\par

\cite{cordeau00} Presents a VRPTW Problem as an extension of the Capacitated Vehicle routing problem, with hard and soft timewindows, where soft can be broken agains a cost and hard time-windows cannot.
They present a multi-commodity network flow formulation of the problem and use approximation methods to derive upper bounds and use Lagrangean relaxation and column generation to derive lower bounds. 
To produce integer solutions they use cutting and branching strategies and finally they also present special cases and extensions. 
\cite{solomon87} formalized an algorithm for general PDP
\cite{lu06} insertion based heuristic
\cite{kalantari85} An algorithm for the travelling salesman problem with pickup and delivery of customers
\cite{dumas91} pickup and delivery problem with time windows
\cite{fagerholt10} speed optimization and reduction of fuel emissions
\cite{berbeglia10} dynamic and static pickup and delivery problems survey and classification scheme 
\cite{berbeglia07} static
\cite{zhou13} Inventory management and inbound logistics optimization
\cite{savelsbergh95} General pickup and delivery problem.
\cite{dror89} Properties and solution framework

\cite{christiansen02} present a ship scheduling PDP problem of bulk cargoes with multiple time windows. 
They also handle specific shipping industry challenges regarding ship idle time, transport risks due to weather and unpredictable service time at ports. 
They use a set partitioning approach to solve the problem and found that you could increase the robustness of the schedules on account of costs.
\cite{christiansen04} Ship routing and scheduling 
\cite{hemmati14} Consider a class of cargo ship routing and scheduling problems from tramp and industrial shipping industry. 
They provide solutions to a wide range of benchmark instances both optimal for smaller instances, using a commercial mixed-integer programming solver.
And they present adaptice large neighbourhood search heuristics to larger instances.
They also provide the benchmark instances and an instance generator.
\cite{hemmati16} A iterative two-phase hybrid metaheuristic for multi-product short sea inventory routing problem
\cite{ferreira18} variable neighbourhood search for vehicle routing problem with multiple time windows
\cite{desrosiers95} Time constrained routing and scheduling


In this paper we will propose a mathematical formulation to a multivehicle PDPTW problem, with certain adjustments to satisfy the needs of an automobile manufacturing company.
To these type of companies the inbound production side of the supply chain is more in focus and require ajdustments on the classical PDPTW.
In these type of industries companies often facing challenges related to the production side of the problem, which is often referred to as looking at the problem from an inbound perspective.
Inbound oriented perspective means that you have less focus on the the supplier side of the of the production, and on the means of transportation as long as what is being transported arrives on time for production.
This changes our problem away from the classical problems mentioned above. 
Instead of having a customer oriented view, we must instead shift our view to focus on getting a set of orders from a set of suppliers to a set of factories where the orders are needed for production. 
\par
Being more inbound oriented also leads to having more focus on the problems that sometimes occur within the factory and the delivery part of the problem.
Some manufacturers might have alot of traffic on their factories, and want to limit that, saying that a vehicle can only visit a certain amount of docks for each visit.
This might lead to more ineffective solutions and more use of vehicles in some cases however, for the inbound oriented producer it is a nessecary limitation. To our knowledge dock constraints has not been researched before in PDPTW problems. \par

Another result from the inbound thinker is that instead of using your own fleet of vehicles, you hire a logistics carrier to do your transports.
The carries might have different cost structures and payment methods which you have no influence over.
Leading you to need a model that takes that into account.
We focus here on dealing with a carrier that offers a varying cost per kilometer.
Meaning that the cost parameter of the standard PDP will change depending on how far a vehicle is travelling. 
Using a carrier also changes the problem to that you dont care so much where the vehicle is travelling from to its first pickup.   
The vehicle rather starts at its first pickup point because the cost of the car manufacturer starts only from the time of the first pickup.
Before pickup and after delivery the carrier has the costs. To our knowledge very little research has been made into using a logistics carrier as vehicle fleet. \par

In car manufacturing business you might have production periods that go over several days with strict opening times on delivery. 
This leads to the problem of multiple time windows.
The manufacturer might only want the parts delivered in the morning on one of three days but have a break in the middle of each time window (lunch). 
There might also be advantageous to move the delivery of one order to the next day to bundle orders together and we want our model to be able to handle such problems. \par
In \cite{favaretto07} they propose a time window model to handle multiple time windows and multiple visits. 
We are using a modified version of this model to handle the multiple time windows in this paper to handle the time window constraints.
However as they focus on satisfying a set of customers instead of production companies our model differs greatly when it comes to the rest of the constraints.\par
\par
To make a model that is as realistic as possible it is important to integrate each of the above aspects and take them all into account. 
One aspect of the model could greatly influence the decision of another so it is important to build a model that satisfies all the aspects above. 
Research often forcuses on solving single issues, like the multiple time windows, to handle this specific problem for a certain industry. 
However handeling all of the above mentioned aspects have to our knowledge never been researched before and makes it therefore an important problem to solve. 
\par
The goal of this paper is to present a realistic mathematical model that can be further developed into a realistic model solving pickup and delivery problems on a daily/weekly basis for manufacturing companies. 
The goal is also to have a mathematical basis to be used in a solver such as AMPL to solve a small instance, and finally to make a heuristic model that can solve the problem efficiently and be used by companies in similar sort of industries.

\section{Problem Formulation} \label{sec:PForm}
Like mentioned above to take the perspective of a vehicle manufacturing company, we see that we have a certain set of factories where the products are produced, each with a set of demanded parts/orders for production.
The parts, or orders (typically named transport orders but we refer to orders here), can be delivered from a set of suppliers.
At a given point in time (could be delivery for one day or a week) we consider a planning problem with a demand for orders that should be satisfied with the given suppliers using vehicles provided by different logistic carriers.
\linebreak


\begin{tabular}{l c l }
              &Indices          			        	\\ 
    $v      $ &-& vehicle     					        \\
    $i      $ &-& node 	        				        \\
    $f      $ &-& factory 		                                \\
    $p      $ &-& timewindow                    		        \\
    $s      $ &-& stop location 				        \\
    $\alpha $ &-& distance interval in cost structure		        \\
    $\beta  $ &-& weight interval in cost structure		        \\
\end{tabular}
\linebreak
\linebreak
\par

The vehicles have different capacities(kg and volume), cost structures, incompatabilities and start at the first pickup location at the first pickup time (ie. costs to get from start to first pickup are not relevant to the manufacturer).
The cost structure depend on the total distance a vehicle is driving, the maximum weight transported by that vehicle, and a fix cost. There is also a cost for stopping at a node. \par
When a delivery is assigned to a vehicle, the vehicle must load the delivery from the supplier, and deliver the delivery at the factory dock.
The docks (each reperesented by a different node) in each factory can differ according to which order is being delivered and there is a limit to how many docks each vehicle can unload at per visit to the factory. \par
Each delivery/pickup can have several time windows, lapsing sometimes over several days. If a car arrives before a time window it has to wait. \par
The mathematical formulation of the problem will now follow. 
The problem can be viewed as a graph $G(E,N)$ where $N=\{0,...,2n\}$ are the verticies and $n$ is the number of orders in the problem, and $E=\{(i,j): i,j \in N, i \neq j\}$ represent the edges in the graph.
\linebreak


\begin{tabular}{l c l }
              &Sets 					        	\\ 
    $N      $ &-& nodes $\{1,..,2n\}$ where $n$ is number of orders     \\
    $V      $ &-& vehicles  					        \\
    $E      $ &-& edges 					        \\
    $E_v    $ &-& edges visitable by vehicle $v$ 		        \\
    $N_v    $ &-& nodes visitable by vehicle $v$  		        \\
    $N^P    $ &-& pickup nodes 					        \\
    $N^D    $ &-& delivery Nodes 				        \\
    $F      $ &-& factories 					        \\
    $N_f    $ &-& delivery nodes for factory $f$ 		        \\
    $A_v    $ &-& index of elements in the cost where $\alpha$ goes     \\
              & & from $(1,..,\gamma_v)$ and $\beta$ from $(1,..,\mu_v)$\\
    $P_i    $ &-& set of time windows at node $i$, $\{1,..,\pi_i \}$	\\
    $T_{i}  $ &-& set of time parameters $[ \underline{T_{ip}},  
		  \overline{T_{ip}} ]$ at node $i$ where $p \in P_i$	\\
    $S      $ &-& set of stops indicating a pickup/delivery location    \\
    $L_s    $ &-& Sets of nodes sharing a stop location 
                  $s \in S$ 	                                        \\
\end{tabular}
\linebreak
\linebreak
\par
The set of vehicles used is denoted by $V$ and weight capacity of each vehicle $v \in V$ is denoted by $K_v^{kg}$ and volume capacity is denoted $K_v^{vol}$.
The set of Edges that each vehicle can traverse is represented by $E_v$. 
Since $n$ is the number of orders in the problem, then if $i$ is a specific pickup-node then $i+n$ corresponds to the delivery node for the same order.
The set of pickup Nodes (supplier docks) we denote using $N^P$ and each delivery node (Factory dock) is denoted by $N^D$. 
All nodes are therefore equivalent to $N = N^P \cup N^D$. 
Each Factory, $f \in F$, also has a set of Nodes belonging to the same factory wich we denote $N_f$. Since all factories are delivery nodes these sets only include delivery nodes.
Each vehicle has a set of Nodes it can travel to represented by $N_v$.
This set also includes an origin node, $o(v)$ and a destination node $d(v)$ which is a fictive start and ending point unique to each vehicle $v$. 
The distance and costs from here to the first pickup is zero.
The factory docking limit is denoted by $H_f$. 
\begin{tabular}{l c l }
    			        &Parameters 							\\ 
    $n    	            $   &-& amount of orders     					\\
    $K_v^{kg}  	            $   &-& weight capacity of vehicle $v\in V$	 			\\
    $K_v^{vol} 	            $   &-& volume capacity of vehicle $v\in V$	 			\\
    $o(v)                   $   &-& starting node of vehicle $v$                                \\
    $d(v)                   $   &-& ending node of vehicle $v$                                  \\
    $Q_i^{kg}  	            $	&-& weight of order at node $i\in N$				\\
    $Q_i^{vol} 	            $	&-& volume of order at node $i\in N$				\\
    $H_f  	            $	&-& docking limit at factory $f\in F$				\\
    $T_{ijv}                $ 	&-& travel time for vehicle $v\in V$ over edge $(i,j)\in E_v$	\\
    $\pi_i	            $	&-& amount of time windows at node $i\in N$			\\ 
    $\overline{T_{ip}}      $   &-& upper bound time of time window $p\in P_i$ at node $i\in N$ \\
    $\underline{T_{ip}}     $	&-& lower bound time of time window $p\in P_i$ at node $i\in N$ \\
    $\gamma_v               $   &-& amount of distance intervals for vehicle $v$                \\
    $\mu_v                  $   &-& amount of weight intervals for vehicle $v$                  \\                   
    $C^{km}_{v\alpha\beta}  $	&-& cost per distance unit (km) in cost matrix element 
                                    $(\alpha, \beta) \in A_v$ for vehicle $v$	                \\
    $C^{kg}_{v\alpha\beta}  $	&-& cost per weight unit (kg) in cost matrix element 
                                    $(\alpha, \beta) \in A_v$ for vehicle $v$	                \\
    $C^{fix}_{v\alpha\beta} $	&-& fixed cost in index $(\alpha, \beta) \in A_v$ for vehicle $v$	\\
    $C^{stop}_i             $   &-& costs of making a stop at node $i$ 	                        \\
    $C_i                    $   &-& cost of not transporting order $i\in N^P$                   \\
    $D_{ij}	            $   &-& distance between node $i\in N$ and  $j\in N$		\\
    $B_\alpha               $   &-& distance for interval $\alpha$ in cost matrix $A_v$ column index\\
    $Z_\beta                $   &-& weight for interval $\beta$ in cost matrix $A_v$ row index    \\ 
\end{tabular}
\linebreak
\linebreak
\par
Each delivery node has a variable $h_{i}$ indicating how many docks have been visited including the node $i$. \par 
Each pickup node has a weight $Q_i^{kg}$ and a volume $Q_i^{vol}$ parameter indicating the weight and volume of the order at that node.
Each node has a set $T_{i}$ of time windows represented by $[ \underline{T_{ip}},  \overline{T_{ip}} ] \in [0,T]$ where $p \in P_i=\{0,1,...,\pi_i\}$ and all nodes should be picked up and delivered within given timewindows.
Each node has a current time based on when its being served, denoted by $t_{i}$ and where $i \in N$. 
The distance from node $i$ to node $j$ is denoted by $D_{ij}$ and the time for each vehicle $v$ to travel between them is represented by $T_{ijv}$. \par 
Each time a vehicle $v$ makes a stop and a node $i$ there will be a stop cost represented by $C_i^{stop}$. 
The costs of vehicle $v$ depends on the total distance of that vehicle and the maximum weight transported. 
Each possible interval of weight and distance is represented by an index pair $(\alpha,\beta)$, where $\alpha$ is the distance interval index ranging $(1..\gamma_v$ and $\beta$ is the weight interval ranging from $(1..\mu_v)$. 
Together these pairs make a matrix we refer to in this paper as a cost matrix. 
Each type of cost has a matrix, including distance, weight, and fix costs and the cost in a certain interval ($\alpha$,$\beta$) is represented by $C_{v\alpha\beta}^{cost-type}$. 
The total distance travelled by vehicle $v$ will be denoted by the variables $d_{v\alpha\beta}$ for each $(\alpha,\beta)\in A_v$, where only one variable per vehicle will have the value equal to the total distance of that vehicle. 
The maximum weight transported by a vehicle is represented by $l_{v\alpha\beta}$ and also only one of these variables per vehicle will have a the corresponding value.
Which $d_{v\alpha\beta}$ and $l_{v\alpha\beta}$ has a value will be determined by the binary variable $b_{v\alpha\beta}$ and the distance insterval parameter $B_\alpha$ and the weight interval parameter $Z_\beta$.  
\linebreak

\begin{tabular} {l c l}
    			&Variables										\\
	$x_{ijv}$	&-& 	binary indicating travel from node $i\in N$ to $j\in N$ of vehicle $v\in V$	\\
	$y_i$	        &-& 	binary indicating that an order $i\in N^P$ is not picked up             	\\
        $l_{iv}^{kg} $ 	&-& 	weight of vehicle $v$ after visiting node $i$        				\\
        $l_{iv}^{vol}$	&-&	volume of vehicle $v$ after visiting node $i$  					\\ 
	$h_i	$	&-&	docking times in factory after visiting node $i\in N_f$				\\
	$t_i 	$	&-&	time after visiting node $i\in N$						\\
	$u_{ip}	$	&-&	binary indicating usage of time window $p\in P_i$ at node $i$			\\
        $d_{v\alpha\beta} $	&-& 	total distance travelled of vehicle $v\in V$ if it fits in interval
                                                                                    $(\alpha, \beta) \in A_v$	\\
        $b_{v\alpha\beta} $	&-&	binary indicating interval $(\alpha, \beta) \in A_v$ for vehicle $v\in V$	\\
        $l_{v\alpha\beta}$ &-&	the highest weight trasported by vehicle $v \in V$ for interval $(\alpha,\beta) \in A_v$	\\

\end{tabular}
\linebreak
\linebreak
\par

$l_{iv}^{kg}$ is the weight and $l_{iv}^{vol}$ is the volume on the vehicle $v$ leaving node $i$. $x_{ijv}$ is a binary vairable indicating if vehicle $v$ is travelling between $i$ and $j$ node. 
The cost of not transporting an order will be represented by $C_i$ for each node $i$, with a corresponding binary variable $y_i$, indicating that an order is not picked up.

\begin{equation}
\label{objectfunc}
        \min\sum_{v\in V} \sum_{(\alpha, \beta) \in A_v} ( C^{km}_{v\alpha\beta}d_{v\alpha\beta} + C^{kg}_{v\alpha\beta}l_{v\alpha\beta} + C^{fix}_{v\alpha\beta}b_{v\alpha\beta} ) + \sum_{v\in V}\sum_{s\in S}\sum_{\substack{i \in L_s\\j \in N_v{\notin} L_s}}C^{stop}_ix_{ijv} + \sum_{i\in N^P}C_iy_i
\end{equation}



subject to:
\begingroup
    \allowdisplaybreaks
\begin{alignat} {2}
    \sum_{v\in V}\sum_{j\in N_v}x_{ijv} + y_i = 1,	        	&\quad\quad\quad&& 	 i \in N^P 				\label{eq:2} 	\\[4pt]
    \sum_{j\in N_v}x_{ijv} - \sum_{j\in N_v}x_{jiv} = 0,		& 	&&	v \in V, i\in N_v \notin \{o(v), d(v)\}		\label{eq:3} 	\\[4pt]
    \sum_{j\in N_v}x_{o(v)jv} = 1,				        & 	&&	v \in V 					\label{eq:4}	\\[4pt]
    \sum_{j\in N_v}x_{j d(v)v} = 1,					& 	&&	v \in V 					\label{eq:5}	\\[4pt]
    \sum_{j\in N_v}x_{ijv} - \sum_{j\in N_v}x_{(i+n)jv} = 0,  	        & 	&&	v \in V, i\in N^P_v 				\label{eq:6}	\\[4pt]
    l_{iv}^{kg} + Q_j^{kg} - l_{jv}^{kg} \leq K_v^{kg}(1-x_{ijv}),      & 	&&	v \in V, j\in N_v^P, (i,j) \in E_v 		\label{eq:7} 	\\[8pt]
    l_{iv}^{kg} - Q_j^{kg} - l_{(j+n)v}^{kg} \leq K_v^{kg}(1-x_{i(j+n)v}),& 	&&	v \in V ,j\in N_v^P, (i, n+j)\in E_v 		\label{eq:8}	\\[8pt]
    0 \leq l_{iv}^{kg} \leq K_v^{kg},					& 	&&	v \in V, i \in N^P_v 				\label{eq:9}	\\[8pt]
    l_{iv}^{vol} + Q_j^{vol} - l_{jv}^{vol} \leq K_v^{vol}(1-x_{ijv}),      & 	&&	v \in V, j\in N_v^P, (i,j) \in E_v 		\label{eq:10} 	\\[8pt]
    l_{iv}^{vol} - Q_j^{vol} - l_{(j+n)v}^{vol} \leq K_v^{vol}(1-x_{i(j+n)v}),&	&&	v \in V ,j\in N_v^P, (i, n+j)\in E_v 		\label{eq:11}	\\[8pt]
    0 \leq l_{iv}^{vol} \leq K_v^{vol},					& 	&&	v \in V, i \in N^P_v 				\label{eq:12}	\\[8pt]
    h_{i} + 1 - h_{j} \leq (H_f+1)(1-x_{ijv}),				& 	&&	v \in V, f\in F,  i \in N_f, j \in N_f, j\neq i	\label{eq:13}	\\[4pt]
    h_{j} \leq H_f,                                                     & 	&&	v \in V,f\in F, j\in N_f,  			\label{eq:14}	\\[4pt]
    h_{j} \geq \sum_{\substack{i\in N_v\\i{\notin} N_f}}(x_{ijv})       & 	&&	v \in V, j \in N_f                      	\label{eq:15}	\\[4pt]
    \sum_{p\in P_i} u_{ip} = 1,						& 	&& 	i \in N 					\label{eq:16}	\\[4pt]
    \sum_{p\in P_i} u_{ip}\underline{T_{ip}} \leq t_{i},		& 	&&	i \in N 					\label{eq:17}	\\[4pt]
    \sum_{p\in P_i} u_{ip}\overline{T_{ip}} \geq t_{i},			& 	&& 	i \in N 					\label{eq:18}	\\[4pt]
    t_{i} + T_{ijv} - t_{j} \leq (\overline{T_{i\pi_i}} + 		& 	&&						\nonumber\\
    T_{ijv})(1 - x_{ijv}),						& 	&& 	v \in V, (i,j) \in E_v				\label{eq:19}	\\[8pt]
    t_{i} + T_{i(i+n)v} - t_{(i+n)} \leq 0,				& 	&& 	v \in V, i \in N_v^P				\label{eq:20}	\\[8pt]
    \sum_{(\alpha,\beta)\in A_v}d_{v\alpha\beta} = 
    \sum_{(i,j) \in E_v} x_{ijv}D_{ij},					& 	&&	v \in V						\label{eq:21}	\\[4pt]
    \sum_{(\alpha,\beta) \in A_v} l_{v\alpha\beta} \geq l_{iv}^{kg}	&	&& 	v \in V, i \in N_v				\label{eq:22}	\\[4pt]
    B_{(\alpha-1)}b_{v\alpha\beta} \leq d_{v\alpha\beta} \leq 
    B_\alpha b_{v\alpha\beta},				        	& 	&&	v \in V, (\alpha, \beta) \in A_v		\label{eq:23}	\\[8pt]
    Z_{(\beta-1)}b_{v\alpha\beta} \leq l_{v\alpha\beta} \leq 
    Z_\beta b_{v\alpha\beta},				        	& 	&&	v \in V, (\alpha, \beta) \in A_v		\label{eq:24}	\\[8pt]   	
    \sum_{(\alpha, \beta) \in A_v} b_{v\alpha\beta} \leq		
    \sum_{j\in N_v}x_{o(v)jv},                                          & 	&&	v \in V						\label{eq:25}	\\[4pt]
    h_i, t_i \geq 0,							& 	&&	i \in N 					\label{eq:26}	\\[8pt]
    u_{ip} \in \{0, 1\},						& 	&&	i \in N, p \in P_i 				\label{eq:27}	\\[8pt]
    b_{v\alpha\beta} \in \{0, 1\},					& 	&&	v \in V, (\alpha,\beta) \in A_v 		\label{eq:28}	\\[8pt]
    d_{v\alpha\beta}, l_{v\alpha\beta} \geq 0                           &       &&      v \in V, (\alpha,\beta) \in A_v 		\label{eq:29}	\\[8pt]
    y_i \in \{0, 1\},                                                   &       &&      i \in N^P                                       \label{eq:30}   \\[8pt]
    x_{ijv} \in \{0, 1\},						&	&&	v \in V, (i, j) \in E_v 			\label{eq:31}
\end{alignat} 
\endgroup

\par
The objective function (\ref{objectfunc}) sums up to the cost of all vehicles given corresponding costs from their cost matrix. Costs could be variable per distance, weight, fixed and/or related to stops made. Loads not transported will be penilized with costs and the aim is to minimize the sum of all these costs.
Constraint \ref{eq:2} secures that a load is picked up once and only by one vehicle or not picked up at all. 
\ref{eq:3} makes certain that if a load at a node j is travelled to it is also left. This is not the case for origin and final destination node.
Then \ref{eq:4} makes sure that origin is only left once by each vehicle, 
and \ref{eq:5} ensures that desination is only arrived to once.
Finally \ref{eq:6} is there to say if a load is picked up it must also be delivered. \par

Regarding weight, constraint \ref{eq:7} ensures that if a load at node $j$ is picked up, the weight before plus the weight of the load at node $j$ minus the weight leaving the node is equal to zero. If not the summation cannot exceed the capacity of the vehicle.
Also \ref{eq:8} makes certain that at delivery of node $j$ the weight before delivery minus the weight of the load at $j$ minus the weight after, is equal to zero, ie. the weight of the vehicle after is lighter than before by exactly the weight of the load at $j$. 
The final weight constraint \ref{eq:9} says that weight leaving node $i$ always has to be between $0$ and the capacity of the vehicle. \par

The next constraints \ref{eq:10}-\ref{eq:12}, ensures the same as for the weight but for volume, that each load is increased by the volume of the order, that the volume is decreased once delivered and that the volume at any node is between $0$ and the volume capacity of that vehicle.

It follows from constrain \ref{eq:13} that within a certain factory, if you travel between two nodes, the amount of stops you have made should always be iterated by one. 
If you are not travelling between them, the summation cannot be bigger than one plus the limitation corresponding to that factory.
The next constraint on factories \ref{eq:14}, ensures that any node visited in a factory cannot exceed the docking limit.
Then \ref{eq:15} makes sure that when a vehicle enters a factory from outside, the docking amount gets an initial value of 1, and if one is not travelling from $i$ to $j$ the value has to be greater than or equal to 0. \par

Constraint \ref{eq:16} ensures only one timewindow is used per node, and \ref{eq:17} says that the time a node $i$ is visited has to exceed a lower timewindow limit.
\ref{eq:18} ensures the same upper bound timewindow is used. 
Then constraint \ref{eq:19} ensures that the travel from one node to the next is appropriately increased by the travel time between them and for any other two nodes the values cannot exceed a large constant.
Finally \ref{eq:20} ensures that the time a vehicle visits the delivery of an order must always be after the pickup of that order. \par

From \ref{eq:21} we have that for each vehicle, the sum of the total distance variables has to be equal to the total travel distance of that vehicle.
Then regarding weight \ref{eq:22} ensures that the sum of the max weight of a vehicle $v$ is greater than or equal to the weight at all nodes visited by that vehicle.
Constraint \ref{eq:23} ensures that for each vehicle $v$ the total distance variable can only exist in the appropriate distance interval.
The same is the case in \ref{eq:24} for maximum weight in the appropriate weight interval.
Finally \ref{eq:25} says that if a vehicle is not leaving its origin node, there cannot be a cost interval binary for that vehicle, which in turn ensures that we dont calculate the fixed costs of said vehicle. \par
Constraints \ref{eq:26} to \ref{eq:31} are ensuring positive and binary variables where appropriate.  \par

\section{Final Remarks}
The model presented in this paper is ready for testing in a solver software such as AMPL. Making a small instance to test the model would then be the next step before adapting the model to an even more complex version of itsself and developing a heuristic to solve the problem. 


\bibliographystyle{abbrvnat}
\bibliography{references}
\end{document}
