\documentclass[a4paper,12pt]{article}
\usepackage[usenames, dvipsnames]{color}
\usepackage{pgfplots}
\usepackage{hyperref}

\begin{document}

\title{Pickup \& Delivery Problem with Mulitiple Time Windows}
\author{Preben Bucher-Johannessen}
\date{\today}
\maketitle

\pagenumbering{roman}
\tableofcontents
\newpage
\pagenumbering{arabic}

\section{Introduction}
TODO..

\section{Problem Formulation} \label{sec:PForm}
Our problem is typically for a vehicle manufacturing company that has a certain set of factories where the vehicles are produced, each with a set of demanded parts for production. The parts, or orders (typically named transport orders but we refer to orders here), can be delivered from a set of suppliers. At a given point in time (could be delivery for one day or a week) we consider a planning problem with a demand for orders that should be satisfied with the given suppliers using vehicles provided by different logistics companies. \par 
The vehicles have different capacities, cost structures, incompatabilities and start at the first pickup location at the first pickup time (ie. costs to get from start to first pickup are not considered neither is time). The cost structure can be multidimentional taking into account total distance and volume in different formats (variable costs, fix costs etc). \par
When a delivery is assigned to a vehicle, the vehicle must load the delivery from the supplier, and deliver the delivery at the factory dock. The docks in each factory can differ according to which order is being delivered and there is a limit to how many docks each vehicle can visit at each factory. \par
Each delivery/pickup can have several time windows, lapsing sometimes over several days. If a car arrives before a time window it has to wait. \par
The mathematical formulation of the problem will now follow where the set of vehicles used is denoted by $V$ and capacity of each vehicle $v \in V$ is denoted by $K_v$. We let $n$ be the number of orders in the problem, so if $i$ is a specific pickup-node then $i+n$ corresponds to the delivery node for the same order. Here it follows that for each vehicle we have a set $O_v$ of orders that vehicle $v \in V$ can transport based on the incompatabilities. 
The set of pickup Nodes (supplier docks) we denote using $N^P$ and each delivery node (Factory dock) is denoted by $N^D$. All nodes are therefore represented by $N = N^P \cup N^D$ Each Factory, $f \in F$, also has a set of Nodes belonging to the same factory wich we denote $N_f$. The delivery nodes that belong to one factory can then be represented as $N_f^D = N_f \subset N^D$. 
Each vehicle has a set of Nodes it can travel to (corresponding to orders $O_v$) represented by $N_v$. This set also includes an origin node, $o(v)$ and a destination node $d(v)$ corresponding to the first pickup and the last delivery of vehicle $v$. The set of Edges that each vehicle can traverse is represented by $E_v$. The factory docking limit is denoted by $D_f$. Each vehicle has a given amount of docks per factory it wants to deliver to represented by $d_{f}^v$. \par 
Each node has a set of $h_i$ time windows represented by $[ \underline{T_{ip}},  \overline{T_{ip}} ] \in [0,T]$ where $p \in \{0,1,...,h_i\}$ and all nodes should be picked up and delivered after time 0 and before time T. Each node has a current time based on when its being served, denoted by $t_{i}$ and a current load, denoted by $l_{i}$, where $i \in 2n$ and $v \in V$. The distance from node $i$ to node $j$ is denoted by $d_{ij}$. The cost of vehicle $v$ driving from node $i$ to $j$ is denoted by $C_{ij}^{vq(v)w(v)}$ and the corresponding time between the nodes $T_{ij}^v$. $l_{i}^v$ is the load of vehicle $v$ leaving node $i$ and $x_{ij}^v$ is a binary vairable indicating if vehicle $v$ is travelling between $i$ and $j$ node. The cost of not transporting an order will be the same for each order (should be set relatively high to avoid dummy transports) so we denote this with $C^N$ and each order, $i$,  not transported is represented by a binary variable $y_i$. 
\begin{equation}
\label{eq:1}
min\sum_{v\epsilon V} \sum_{(i,j) \epsilon E_v} c_{ijv}d_{ij}x_{ijv} + \sum_{i\in N^P}C_iy_i
\end{equation}

subject to:
\begin{equation} \label{eq:3}
    \sum_{v\epsilon V}\sum_{j\epsilon N_v}x_{ijv} + y_i >= 1, ~~~~~ i \in N^P
\end{equation}

\begin{equation} \label{eq:4}
    \sum_{j\epsilon N_v}x_{ijv} - \sum_{j\epsilon N_v}x_{jiv} = 0, ~~~~~ v \in V, i\in N_v \backslash \{o(v), d(v)\} 
\end{equation}

\begin{equation} \label{eq:5}
    \sum_{j\epsilon N_v}x_{o(v)jv} = 1, ~~~~~ v \in V
\end{equation}

\begin{equation} \label{eq:6}
    \sum_{j\epsilon N_v}x_{j d(v)v} = 1, ~~~~~ v \in V
\end{equation}

\begin{equation} \label{eq:7}
    \sum_{j\epsilon N_v}x_{ijv} - \sum_{j\epsilon N_v}x_{(i+n)jv} = 0, ~~~~~ v \in V, i\in N^P_v
\end{equation}

\begin{equation} \label{eq:8}
    l_{iv} + Q_j - l_{jv} \leq K_v(1-x_{ijv}),~~~~~~~~ v \in V,j\in N_v^P,(i,j) \in E_v
\end{equation}

\begin{equation} \label{eq:9}
    l_{iv} - Q_j - l_{jv} \leq K_v(1-x_{i(j+n)v}), ~~~~~~~~ v \in V,j\in N_v^P, (i, n+j)\in E_v
\end{equation}

\begin{equation} \label{eq:10}
0 \leq l_{iv} \leq K_v, ~~~~~~~~ v \in V, i \in N^P_v
\end{equation}

\begin{equation} \label{eq:11}
    h_{i} + 1 - h_{j} \leq H_f(1-x_{ijv}), ~~~~~~~~ v \in V, (i,j)\in E_V, i\in N^D_f, j\in N^D_f, f\in F
\end{equation}

\begin{equation} \label{eq:12}
    h_{i} \leq H_f, ~~~~~~~~ v\in V, i\in N_f^D, f\in F
\end{equation}

\begin{equation} \label{eq:13}
    h_{i} = x_{ijv}, ~~~~~~~~ v \in V, (i, j)\in E_V, j \notin N_f^D, i \in N^D_f, f \in F
\end{equation}

\begin{equation} \label{eq:15}
    \sum_{p\in P_i} u_{ip} = 1, ~~~~~~~~ i\in N
\end{equation}

\begin{equation} \label{eq:15}
    \sum_{p\in P_i} u_{ip}\underline{T_{ip}} \leq t_{i}, ~~~~~~~~ i\in N
\end{equation}

\begin{equation} \label{eq:15}
    \sum_{p\in P_i} u_{ip}\overline{T_{ip}} \geq t_{i}, ~~~~~~~~ i\in N
\end{equation}

\begin{equation} \label{eq:15}
    t_{i} + T_{ijv} + w_j - t_{j} \leq (\overline{T_{\pi_i i}} + T_{ijv} + w_j)(1 - x_{ijv}), ~~~~~~~~ v \in V, (i,j) \in E_v
\end{equation}

\begin{equation} \label{eq:18}
    \sum_{r \in R}l_{vr} \geq l_{iv} + Q_j,~~~~~~~~ v \in V, j\in N_v^P, (i,j)\in E_v
\end{equation}

\begin{equation} \label{eq:19}
    B_{(s-1)}b_{vs} \leq d_{vs} \leq B_sb_{vs}
\end{equation}

\begin{equation} \label{eq:20}
 \sum_{s \in S} b_{sv} = 1 ,~~~~~~~~ v \in V
\end{equation}


\begin{equation} \label{eq:18}
    \sum_{s \in S}d_{vs} = \sum_{(i,j) \in E_v} x_{ijv}d_{ij},~~~~~~~~ v \in V
\end{equation}

\begin{equation} \label{eq:19}
    B_{(s-1)}b_{vs} \leq d_{vs} \leq B_sb_{vs}
\end{equation}

\begin{equation} \label{eq:20}
 \sum_{s \in S} b_{sv} = 1 ,~~~~~~~~ v \in V
\end{equation}

\begin{equation} \label{eq:20}
    \sum_{s \in S} c_{sv}d_{sv} = \sum_{(i,j) \in E_v} c_{ijv}d_{ij}x_{ijv} ,~~~~~~~~ v \in V
\end{equation}

\begin{equation} \label{eq:14}
    c_{ijv} = c_{klv}, ~~~~~~~~ (i, j) \in E_v, (k, l) \in E_v
\end{equation}


\begin{equation} \label{eq:14}
0 \leq h_{i}, ~~~~~~~~ i \in N
\end{equation}

\begin{equation} \label{eq:21}
y_i \in \{0, 1\} ~~~~ i \in N^P
\end{equation}

\begin{equation} \label{eq:22}
x_{ijv} \in \{0, 1\} ~~~~ v \in V, (i, j) \in A_v
\end{equation}




\par
The objective function (\ref{eq:1}) sums up to the cost of the spot cars and the aim is to minimize these costs. \newline \newline


List of Variables and parameters: \newline
$v \in V$ - Vehicles \newline
$n$ - number of orders\newline
$s \in S$ - size of cost structure regarding distance \newline
$i$ - orders\newline
$K_v$ - Vehicle Capacity \newline
$l_{iv}$ - Load at node $i$ with vehicle $v$ \newline
$O_v$ - Set of orders $i$ that vehicle $v$ can transport \newline
$N^P$ - pickup nodes\newline
$N^D$ - delivery Nodes\newline
$N^D_f$ - Delivery nodes for factory f \newline
$E_v$ - Edges/Arcs the vehicle $v$ can traverse\newline
$f \in F$ - Factory\newline
$H_f$ - Factory docking limit\newline
$h_{i}$ - factory visits at node $i$  \newline
$T_{ip}$ - Node $i$ has a sets of $p$ time windows $[ \underline{T_{ip}},  \overline{T_{ip}} ]$ where $p \in P_i =\{1,..,\pi_i \}$ \newline
$T_{ijv}$ - Travel time from $i$ to $j$ using $v$\newline
$u_{ip}$ - binary variable for timewindow $p\in P_i$ at node $i$\newline
$M$ - Arbitrary large constant \newline
$w_i$ - Waiting time at node $i$ \newline
$Q_j$ - weight of order at node j \newline
$x_{ijv}$ - Binary variable indicating travel from $i$ to $j$ by vehicle $v$ \newline
$y_i$ - binary indicating order not picked up \newline
$d_{ij}$ - distance between node $i$ and $j$\newline
$r \in R = \{1,...,\alpha\}$ - cost structure for vehicle $v$ weight dimension\newline
$l_{vr}$ - weight of vehicle $v$ and fitting in cost matrix part $r$ \newline
$A_r$ - interval weight limit $r$ of costmatrix\newline
$a_{vr}$ - binary variable determining interval $r$ in cost structure of vehicle $v$\newline 
$s \in S = \{1,...,\beta_v\}$ - cost structure in distance dimention \newline
$d_{vs}$ - the total distance for vehicle $v$ fitting in the cost structure $s$ \newline
$c_{vs}$ - Cost per km for vehicle $v$ in cost structure $s$ \newline
$b_{vs}$ - binary indicating which interval vehicle $v$ is in cost structure $s$ \newline
$B_s$ - distance determening interval $s$ in cost structure
\section{Solution in AMPL}
TODO.. 

\section{References} \label{sec:Ref}

\href{http://www-bcf.usc.edu/~maged/publications/MultiplePickup.pdf}{Lu 2012} \newline
\url{https://www.sciencedirect.com/science/article/pii/S0305054813002165} \newline
\url{https://www.sciencedirect.com/science/article/pii/S0305048304001550} \newline
\url{https://link.springer.com/article/10.1007/s10732-005-5432-5} \newline
\url{https://pubsonline.informs.org/doi/abs/10.1287/trsc.1050.0135} \newline
\href{https://onlinelibrary.wiley.com/doi/epdf/10.1002/nav.10033}{M. Christiansen \& K. Fagerholt (2002) - Robust Ship Scheduling with Multiple Time Windows - } \newline




\end{document}
